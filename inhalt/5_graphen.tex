\section{Graphentheorie}
\subsection*{Gerichteter Graph}
$G=(V,E)$ wobei $V$ Menge aller Knoten z.B. $V=\{v_0,v_1,v_2,\dots,v_n\}$ und $E\subseteq V\times V$ Menge aller Kanten mit $e=(v,u)$. Hierbei steht $v$ für den Startknoten von $e$ und $u$ ist Endknoten von $e$.\\
\emph{Hinweis:}\\
Ist die Kantenmenge $E$ symmetrisch ($(u,v)\in E\wedge (v,u)\in E$) sprechen wir von einem ungerichteten Graphen. In solchen werden keine Schlingen betrachtet.
\subsection*{Adjazente Knoten}
Zwei Knoten, die in einem Graphen durch eine Kante verbunden sind, heißen \emph{adjazent} oder \emph{benachbart}.
\subsection*{Endlicher Graph}
Ein Graph $G$ heißt endlich, wenn die Knotenmenge $V$ endlich ist.
\subsection*{Nullgraph (vollst. unverbunden)}
$G=(V,\emptyset)\Rightarrow$ ohne Kanten
\subsection*{Vollständiger Graph}
$G=(V,V\times V)$ ist vollständig (heißt auch $K_n$ wegen $n$ Knoten) und hat Maximalzahl von $n^2$ Kanten $\Rightarrow$ gerichtet und mit Schlingen. Der Ungerichtete $K_n$ hat $\frac{n\cdot (n-1)}{2}$ Kanten, wobei $n$ die Zahl der Knoten ist.\\
\emph{Beispiel:}\\
\begin{tikzpicture}[scale=.60,transform shape]
\graph [simple,nodes={myblue}, edges={myblue!80, semithick}] {subgraph K_n [n=3, clockwise];};
\end{tikzpicture}
\begin{tikzpicture}[scale=.60,transform shape]
\graph [simple,nodes={myblue}, edges={myblue!80, semithick}] {subgraph K_n [n=4, clockwise];};
\end{tikzpicture}
\begin{tikzpicture}[scale=.60,transform shape]
\graph [simple,nodes={myblue}, edges={myblue!80, semithick}] {subgraph K_n [n=5, clockwise];};
\end{tikzpicture}
\begin{tikzpicture}[scale=.60,transform shape]
\graph [simple,nodes={myblue}, edges={myblue!80, semithick}] {subgraph K_n [n=6, clockwise];};
\end{tikzpicture}
\subsection*{Ungerichteter Graph}
Ein Graph $G=(V,E)$ ist ungerichtet $\Leftrightarrow E$ ist symmetrisch $\Leftrightarrow (u,v)\in E\wedge (v,u)\in E$. Da hier die Kanten ungerichtet, kann Mengenschreibweise verwendet werden.
\subsection*{Schlinge}
Kante mit gleichem Start- und Endknoten. Wird bei ungerichteten Graphen nicht betrachtet.
\subsection*{Bipartite Graphen}
Ein ungerichteter Graph ist bipartit, wenn die Knotenmenge $V$ in zwei disjunkte Teilmengen $U,W$ zerlegbar ist, sodass alle Kanten $e\in E$ einen Endpunkt in $U$ und einen anderen in $W$ haben.\\
\emph{Beispiel:}\\
\begin{tikzpicture}[scale=.75,transform shape]
\graph [simple,nodes={myblue}, edges={myblue!80, semithick}] {subgraph K_nm [n=2,m=2];};
\end{tikzpicture}
\begin{tikzpicture}[scale=.75,transform shape]
\hspace{.3cm}
\graph [simple,nodes={myblue}, edges={myblue!80, semithick}] {subgraph K_nm [n=3,m=1];};
\end{tikzpicture}
\begin{tikzpicture}[scale=.75,transform shape]
\hspace{.6cm}
\graph [simple,nodes={myblue}, edges={myblue!80, semithick}] {subgraph K_nm [n=3,m=2];};
\end{tikzpicture}
\begin{tikzpicture}[scale=.75,transform shape]
\hspace{1cm}
\graph [simple,nodes={myblue}, edges={myblue!80, semithick}] {subgraph K_nm [n=3,m=3];};
\end{tikzpicture}
\subsection*{Eulersche Graphen}
$G$ heißt eulerscher Graph, falls es in $G$ einen geschlossenen Weg gibt, der jede Kante von $G$ enthält.\\
$G$ ist eulerscher Graph $\Leftrightarrow$ jede Ecke von $G$ hat geraden Grad und $G$ ist zusammenhängend.
\subsection*{Untergraph}
Seien $G=(V_G,E_G)$, $H=(V_H,E_H)$ zwei Graphen. $H$ heißt Teilgraph von $G$, wenn $V_H\subseteq V_G$ und $E_H\subseteq E_G$
(wenn also jede Kante von $H$ auch zu $G$ gehört.)\\
\emph{Hinweis:}\\
Der Nullgraph $O_n$ ist Teilgraph jedes Graphen mit $n$ Knoten. Außerdem ist jeder Graph Teilgraph des vollständigen Graphen $K_n$.
\subsection*{Induzierte Teilgraphen}
Sei $G=(V,E)$ ein Graph. Ist $V'\subseteq V$ eine Teilmenge der Knotenmenge $V$ von $G$, dann ist der Untergraph oder
der durch $V'$ induzierte Teilgraph von $G$ der Graph $G[V']=(V',E')$ mit $E'=\{(u,v)\mid u,v\in V'\wedge (u,v)\in E\}$\\
\emph{Beispiel:}\\
Ist $G$ ein Graph hat $G[\{2,3,4\}]$ nur die Knoten $2$, $3$ und $4$ und die entsprechenden Kanten.
\subsection*{Grad eines Knoten}
Der Ausgrad von $v$ ist die Zahl der Kanten, die $v$ als Startknoten besitzen.
Der Ingrad von $v$ ist die Zahl der Kanten, die in $v$ enden.
Ist $G$ ungerichtet stimmen Ausgrad von $v$ und Ingrad von $v$ überein und wird Grad von $v$ genannt.\\
\emph{Hinweis:}\\
Sei $G=(V,E)$ gerichteter Graph, dann gilt $\sum_{v\in V} indeg(v)=\sum_{v\in V} outdeg(v)=|E|$.
Ist $G$ ungerichtet, dann gilt $\sum_{v\in V} deg(v)=2\cdot |E|$.
\subsection*{Wege}
Ein Weg von den Knoten $u$ nach $v$ ist eine Folge benachbarter Knoten. Die Länge
eines Weges ist die Anzahl der Kanten. Ein Weg der Länge $0$ wird als trivialer Weg bezeichnet und besteht nur aus einem Knoten.\\
\emph{Hinweis:}\\
Ein Weg heißt geschlossen, wenn seine beiden Endknoten gleich sind.
\subsection*{Graphzusammenhang}
Knoten $u$ und $v$ eines ungerichteten Graphen heißen zuammenhängend, wenn es
einen Weg in $G$ von $u$ nach $v$ gibt.
\subsection*{Zusammenhangskomponente}
Ein Graph $G$ heißt zusammenhängend wenn jedes Knotenpaar aus $G$ zusammenhängend ist.\\
\emph{Hinweis:}\\
Die Äquivalenzklassen (zusammenhängende Teilgraphen) einer Zusammenhangsrelation $Z$ über einem ungerichteten Graphen $G$ heißen Zusammenhangskomponenten (ZK) von $G$.
\subsection*{Pfade, Kreise}
Als \emph{Pfad} werden Wege in einem Graphen bezeichnet, bei denen keine Kante zweimal durchlaufen wird.
Ein geschlossener Pfad heißt \emph{Kreis}. Bei einem \emph{einfachen Pfad} wird kein Knoten mehrfach durchlaufen.
Ein geschlossener Pfad, der mit Ausnahme seines Ausgangspunktes einfach ist, heißt \emph{einfacher Kreis}.
Ein einfacher Kreis durch sämtliche Knoten des Graphen, heißt \emph{Hamiltonscher Kreis}.
\subsection*{Hamiltonscher Kreis}
Kann der Zusammenhang eines Graphen $G$ durch die Entnahme eines einzigen Knotens (und
sämtlicher mit diesem Knoten benachbarter Kanten) zerstört werden, dann besitzt $G$ keinen
Hamiltonschen Kreis.
\subsection*{Isomorphe Graphen}
Zwei Graphen heißen isomorph zueinander, wenn sie strukturell gleich sind.\\
\emph{Beispiel:}\\
\begin{tikzpicture}[scale=.62,transform shape]
\hspace{.5cm}
\graph [simple,nodes={myblue}, edges={myblue!80, semithick}] {subgraph K_n [n=5, clockwise];
1 -!- 2;
2 -!- 3;
3 -!- 4;
4 -!- 5;
5 -!- 1;
};
\end{tikzpicture}
\begin{tikzpicture}[scale=.62,transform shape]
\hspace{.75cm}
\graph [simple,nodes={myblue}, edges={myblue!80, semithick}] {subgraph K_n [n=5, clockwise];
1 -!- 4;
1 -!- 3;
5 -!- 2;
5 -!- 3;
4 -!- 2;
};
\end{tikzpicture}

\begin{tabular}{c|c|c|c|c|c}
$v$ & $1$ & $2$ & $3$ & $4$ & $5$ \\ 
$\phi (v)$ & $1$ & $4$ & $2$ & $5$ & $3$ \\
\end{tabular}
\subsection*{Komplementäre Graphen}
Sei $G=(V,E)$ ein Graph. Dann ist $\bar{G}=(V,(V\times V)\setminus E)$ der Komplementärgraph von $G$.\\
\emph{Hinweis:}\\
Ein Graph heißt selbstkomplementär wenn $G$ und $\bar{G}$ isomorph sind.
\break



\\
