\\ 
\section{Beweise}
\subsection*{Deduktiv}
Axiome->mittels Regeln/Def.-->Aussagen ableiten->Log.Beweisen.
\subsection*{Induktiv}
Wissen aus Beob./Exp.. vom einz. aufs Ganze.
\subsection*{Direkter Beweis}
Beim direkten Beweis wird Schritt für Schritt mittels \emph{Wenn, Dann} bewiesen. Bsp. Aus $A folt B$, aus $B folgt Z$. und weil Transitivität der Implikation. Aus $A folgt Z$.
Beweisen mittels Wahrheitstabellen.\\
Gegeben: Wahre Aussagen und Werkzeige. Zu zeigen: dass Z gilt. Methode: Folge von Implikationen finden, von A bis Z. Aus Gültigkeit von A, kann Gült. von Z geschlossen werden.
\subsection*{Kontraposition}
Da $p\Rightarrow q\equiv \neg q\Rightarrow \neg p$ kann man die Aussage auch mittels Kontraposition beweisen.
z.B bei ungeraden Zahlen.
\subsection*{Widerspruch, Indirekter Beweis}
Beim Widerspruchsbeweis wird Gegenteil angenommen und in einen Widerspruch geführt.
Also muss die ursprüngliche Aussage wahr sein.\\

\section{Vollständige Induktion}
Ist ein deduktives Argument (weil vollständig, d.h. für jeden Einzelfall auf einen allg. Fall schliessen. 
(deduktiv: die allg. Richtigkeit ableiten).

\subsection{Induktionsbeweis}
Bsp. für n \in \mathbb{N}_0$ gilt: \sum_{i=0}^n 2^{i} = 2^{n+1}-1

\textbf{1.Verankerung} 
Zeigen, dass \textit{A(n$_{0}$)} gilt. Der erste Dominostein muss fallen!\\
$n_{0}$ = 0, dann gilt für A(0):\\
\sum_{i=0}^0 2^{i} = 2^{0+1}-1\\
= 2^{0} = 2-1\\ 
= 1 = 1. $ korrekt, Aussage ist für $ n_{0}=0$ korrekt.\\
\textbf{2.Schritt}
Wir nehmnen an, dass \textit{A(n)} gilt unter der Annahme ($n\in \mathbb{N}$) mit n\geq \ n$_{0}$. Somit sollte also auch A(n+1) gelten. Formal:\\
\forall	$$n$\in \mathbb{N}$ $mit n$\geq \ n$_{0}$ : $A(n)$ $\rightarrow A(n+1)$.\\
Achtung!! manchmal auch umgekehrt:!!\\
\forall	$$n$\in \mathbb{N}$ $mit n$\geq \ n$_{0}$ : $A(n-1)$ $\rightarrow A(n).\\
$\textbf{2.1 Annahme}: $A(n)$\\
'(n)' einsetzen, Formel hinschreiben (oft urspüngliche).\\
\sum_{i=0}^n 2^{i} = 2^{n+1}-1\\
$\textbf{2.2 Behauptung} $A(n+1)$\\
'n+1' bei Annahme einsetzen.\\
\sum_{i=0}^{n+1} 2^{i} = 2^{(n+1)+1}-1\\
($n \Longrightarrow n+1$): Zu zeigen ist also $n+1$ einsetzen $\Rightarrow$ Aussage gilt auch,
\emph{mit Benutzung von Induktionsbehauptung}.\\
\textbf{3. Beweisen} \\
In deisem Bsp.\\
Beahuptung links = Annahme links + 'eins mehr')
n+1 = (n) + (1 mehr)\\
\sum_{i=0}^n 2^{i} = [\sum_{i=0}^n 2^{i}] + 2^{n+1} $|^{n+1}$ für 2^i$\\
= $[2^{n+1}-1] + 2^{n+1} $| [] = Annahme rechts.\\
= $2^1(2^{n+1}-1) $| vereinfachen\\
= $2^{n+2}-1. $\square$ | entspricht  Behauptung rechts.\\

\subsection{Ungleichungen}
Bsp. Ungleichung von Bernoulli\\
$x>-1, n \in \mathbb{N}_0: (1+x)^n \ge 1+n*x.$\\
\textbf{1. Verankerung}: n_0 =0$, dann A(0) = \\
(1+x)^0 \ge 1+0*x. \\
1 \ge 1. $korrekt.\\
\textbf{2.1 Annahme}: A(n)\\
 (1+x)^n \ge 1+n*x, \forall	$$n$ > -1.\\
\textbf{2.2 Behauptung}:A(n) -> A(n+1)\\
 (1+x)^{n+1} \ge 1+(n+1)*x, \forall	$$n$ > \-1.\\
\textbf{3. Beweis}:\\
Behauptung li = Annahme li +'eins'\\
(1+x)^{n+1} = [(1+x)^n] * (1+x)^1\\
\ge (1+n*x)*(1+x)  $|[] Annahme einsetzen.\\
=1+x+nx+nx^2\\
=1+(n+1)x+nx^2 $|wobei nx^2 \ge 0!\\
\ge 1 +(n+1)x. $\square$ | entspricht  Behauptung rechts.\\
\subsection{Allg. Aussagen}
z.B 8^n-1 $ist durch 7 teilbar, für n \in \mathbb{N}$\\
\textbf{1. Verankerung}: n_0 =1$,
dann A(1) = 8^1$-1=7. korrekt.\\
\textbf{2.1 Annahme}: A(n)\\
8^n-1 $ist durch 7 teilbar.\\
\textbf{2.2 Behauptung}:A(n) -> A(n+1)\\
$8^{n+1}-1$ ist durch 7 teilbar.
\textbf{3. Beweis}:\\
Behauptunng li = Annahme li +'eins'\\
$8^{n+1}-1 = 8^1*[8^n-1]$\\
kleiner Trick, rechte Seite -7+7.\\
$= 8^1*8^n-1-7+7\\
= 8^1*8^n-8+7\\
=8* (8^n-1) + 7$\\
$(8^n-1)$ ist durch 7 teilbar, 7 ist durch 7 teilbar, $8*(8^n-1)$ ist durch 7 teilbar. und so auch der ganze Ausdruck.\\
\\

\section{Rekursion}
Idee Induktion: Startelement und dann Nachfolgeelemente. Idee Rekursion: \textbf{Umkehrung der Idee}, Starten irgendwo und gehe zurück bis zum Anfang (wo Funktionswert einfach berechnet werden kann).
Vorgehen ist rekursiv, wenn \textbf{Basisfall} (Basiselement der Menge) beschrieben ist durch \textbf{Anzahl Regeln} (wie man von belieb. Element zu Basisfall kommt).\\
\textbf{Fraktale, Sirpinski-Dreieck, Fibonacci, Fakultät...}
\textbf{Oft: Aus rek. Darstellung die expliz. Darstellung bestimmen!}\\

\subsection{Grundbegriffe Zahlenfolgen}
Einfachtse rekursive Struktur.\\
\textbf{A) Funktionale Definition, Explizite Def.}
Eine (reelle) Zahlenfolge ist eine Funktion: \\
$f:\mathbb{N}-->R$, $n -->f(n)$\\
'für jedes n gibt es einen funkt. Wert.'\\
andere Schreibweisen: $f(n)_{n \in \mathbb{N}}$\\
f(n) allein = \textbf{explizite Definition}.\\
\textbf{B) Rekursive Darstellung, k-ter Ordnung}
1.Ordnung = 1 Startwert, 2.Ordnung=2 Startwerte etc.\\
$f:\mathbb{N}-->R$, werden die:\\
-\textbf{Folgeglieder} (und {(Anfangswerte)}) angegeben: f(1), f(2), ..f(k)..\\
-für $n \ge k+1$ eine \textbf{Vorschrift}(Berechnungsvorschriften), wie das \textbf{n-te Folgeglied f(n)} aus den Folgegliedern f(n-1), f(n-2),..f(n-k) berechnet wird.\\
\textbf{C) Aufzählende Darstellung}\\
Auflistung Folgeglieder: f(1),f(2),..f(n).\\

\textbf{Beispiel Fakultät:}\\
\textbf{C)} 1,2,6,24,120..\\
\textbf{A)} $f:\mathbb{N}-->R$, $n -->f(n): n!$\\
=n*(n-1)*(n-2),...*3*2*1.\\
\textbf{B)} 1. Ordnung $f(1):=1, \\ 
f(n):n*f(n-1)$ für $n \in \mathbb{N}$ mit $n\ge2$\\

\textbf{Beispiel Fibonacci}\\
\textbf{C)} 1,1,2,3,5,8,13,21,34,55.\\
\textbf{A)} $f(n)=1/\sqrt5 ((1+\sqrt5)/2)$....komplex\\
\textbf{B)} 2. Ordnung $f(1):=1$,$f(2):=1$ \\ 
$f(n):= f(n-1)+f(n-2)$ für $n \in \mathbb{N}$ mit $n\ge3$.\\

\textbf{Beispiel explizit, rekursiv}\\
\textbf{A)} 1,4,7,10,13,16,..\\
\textbf{C)} $f(1)=1, f(n):=f(n-1)+3, \forall >1$.\\
\textbf{B)} $f(n)=3*n-2, \forall n \in \mathbb{N}$.\\

\textbf{A)} 2,4,8,16,32,64,..\\
\textbf{C)} $f(1)=2, f(n):=f(n-1)*2, \forall >1$.\\
\textbf{B)} $f(n)=2`n, \forall n \in \mathbb{N}$.\\

\textbf{A)} 3,33,333,3333,..\\
\textbf{C)} $f(1)=3, f(n):=f(n-1)*10+3, \forall >1$.\\
\textbf{B)} $f(n)=\sum_{i=1}^n 3*10^{i-1}, \forall n \in \mathbb{N}$\\

\subsection{Rekursionsbaum}
Berechnung des n-ten Folgeglieds einer rek. Zahlenfolge.\\
gerichteter baum it 'Wurzel' f(n).\\
1. Anfangswerte als Blätter.\\
2. Aufsteigend für jeden Vorgänger gemäss dem Funktionswert ausrechnen und hinschreiben. Bis zu Wurzel f(n).\\

\Tree [.Wurzel [.B [.C Blatt ] [.D Blatt ] ].B [.E {Blatt} ] ]\\
Zuerst f(x) überall hinschreiben, danach ausrechnen, und danaben hinschreiben.
Bsp. 'Fibonacci f(5)'\\
Wurzel=f(5)=5.\\
B=f(4)=3\\
E=f(3)=2 && f(2)=1.\\
C=f(3)=2 && f(2)=1.\\
..
Blätter: f(2)=1 && f(1)=1\\
\\
\\
\\

\subsection{Induktives Schliessen, rek --> expl.}
Mittels Zahlenbeispielen. Viele Varianten aufschreiben.\\
Bsp. f(1), f(n) =4*f(n-1) + 4^{n-1}, für $n \in \mathbb{N}$ mit $n\ge2$\\
\textbf{0. Vermutung}:\\
$f(1) = 1. (=4^0*1)$\\
$f(2)=4*f(1) + 4^1 = 4*1+4 =8$\\
$f(3)=4*f(2) + 4^2 = 4*8+16 =48$\\
$f(4)=4*f(3) + 4^3 = 4*48+64 =256.$\\
Vermutung: $4^{n-1}*n = n*4^{n-1}$.\\
Die Vermutung muss nun bewiesen werden. z.B. mittels vollständiger Induktion.\\
\textbf{1. Verankerung}: $n=1, f(1)=1.$ \\
$4^{1-1}*1 = 1*1=1$. korrekt.\\
\textbf{2.1 Annahme}: $f(n-1)$\\
$f(n-1)=(n-1)*4^{n-2}$
!!Auf diese Art, weil wir annehmen, dass Formel unserer Vermutung für n gilt.\\
\textbf{2.2 Behauptung}:$f(n) = n*4^{n-1}$\\
\textbf{3. Beweis}: n$_{0}$ : $A(n-1)$ $\rightarrow A(n)\\
$f(n) = 4* [f(n-1)] +4^{n-1}$ | rek. Def.\\
$ = 4* [(n-1)*4^{n-2}] +4^{n-1}$ | Annahme[] einsetzen\\
$ = 4^{n-1}(n-1)+4^{n-1}$| Pot.r, umform.\\
$ = 4^{n-1}(n-1+1)$| ausklamm.\\
$ = 4^{n-1}*n.$\square$ \\

\subsection{Beweisen durch Einsetzen}
Vereinfachter/schneller Beweis.\\
1. überprüfen, ob expl. Darstellung für n=1 gilt. (Anfangsbedingung)\\
2. überprüfen, expl. Darstellung einsetzen in rek. Darstellung = korrekte Gleichung?\\
Bsp. explizit: $f(n) = n*4^{n-1}$ für \\
rekursiv: $f(1), f(n)= 4*f(n-1)+4^{n-1}$ für $n\in \mathbb{N}$ mit $n\ge2$\\
\textbf{1. Verankerung}: $f(1) = 1*4^{1-1} = 1$. korrekt.\\
\textbf{2.2 Behauptung durch einsetzen}: $f(n)=n*4^{n-1}$ explizit\\
expl. li in rek.: $n*4^{n-1}$\\
expl. re in rek.: $4*(n-1)*4^{n-2}+4^{n-1}$\\
$ = (n-1)*4^{n-1}+4^{n-1}$\\
$ = n*4~{n-1}+4^{n-1}$\\
$ = n * 4^{n-1}. $\square$\\

\subsection{Unterscheidungen}
$f(n) = s(n) + \sum_{i=1}^k c_{n-1}(n) * f(n-i) $
ist eine lin. Rek.gleich. k-ter Ordnung.\\
\textbf{Typ}: linear | nicht Bruc, nicht expo.\\
\textbf{Ordnung}: k-ter | wie viele f(n-1), f(n-2), f(..)\\
\textbf{Störterm}: s(n) | konstanter Störterm.\\
--> wenn Störterm = 0, dann ist \textbf{homogen}.\\
\textbf{Koeffizientenfunkton}: $c_{n-1}(n)$ |  weil von n abhängig.\\
Ich schreibe: homogene lineare Rek.gleichung 1. Ordnung mit Koeffizientenfunkton c = x.\\

\subsection{Iterationsmethode}
für homogene lin. Rekursionsgleichungen. Ähnlich ind. Schliessen.\\
\textbf{Gegeben}: $f(n) = (n) * f(n-1)$ mit $f(1) = a (a \in  \mathbb{R})$\\
\textbf{Vorgehen}: f(n-1), f(n-2)..f(1) nach und nach einsetzen und versuchen, das Ergebnis zu vereinfachen. Danach beweisen durch vollst. Induktion oder Einsetzungsmethode\\
\textbf{falls c konstant, homogen und 1. Ordnung}: falls c(n)=c eine konst. Funktion ist, dann Lösung:
$f(n) = a * c^{n-1}$ | a = Anfangswert.\\

\section{Rekursionsgleichungen lösen}
bisher, homogen, lin. Rek. 1. Ordnung.\\
\subsection{Inhomogen, lin. Rek., 1. Ordnung}
Oft auch mit Iterationsmethode. Oder mittels \textbf{Lösungsformel}:
Bsp. $f(n) = s + \sum_{i=1}^k c_{n-i} * f(n-i) $\\
$s \in \mathbb{R} , c_{n-i} \in \mathbb{R}, i \in {1,2,..k} $\\ heisst (inhomogene) lineare Rekgleichung k-ter Ordnung mit konst. Koeffizenten.\\
\textbf{Lösungsformel II}: $f(n) = c * f(n-1) + s$ \\
\textbf{Lösung (inhom., lin. Rek.gleich. 1. Ordnung, c konstant)}: Bsp. $f(n) = c * f(n-1) + s$, mit f(1)= a ist: \\
c nicht gleich 1: $f(n) = a * c^{n-1} + \frac{1-c^{n-1}}{1-c} * s $\\
c gleich 1: $a + (n-1)* s$\\

\textbf{Lösung (inhom., lin. Rek.gleich. 1. Ordnung, c konstant, Störterm konstant},Bsp. $f(n) = s(n) + c * f(n-1) + s$, mit f(1)= a ist: \\
f(n) = $all. Lösung hom. Gleichung  + p(n)\\ 
= d * c^{n-1} + p(n)$\\
p(n) = irgendeine spez. Lösung (\textbf{Partikuläre Lösung}) der inhomogenen Rekursion, welche zudem \textbf{nicht} Lösung der zugehörigen hom. Gleichung ist.\\
\textbf{Partikuläre Lösung}: \\
Wie finden? Die noch freien Parameter/Polynomkoeff.werden bestimmt indem der Ansatz p(n) in die Rekursionsgleichung eingesetzt wird.\\
-$Störterm s(n)$: Polynom in n vom Grad m--> $Ansatz$: Allg. Polynom in n vom Grad m.\\
-$s(n): u^n, u \in \mathbb{R}$--> Ansatz: $t * u^n$\\
-$s(n)$: (Polynom in n vom Grad m) $* u^n, u \in \mathbb{R}$ --> Ansatz: (Allg. Polynom in n vom Grad m) * u^n$\\
 \textbf{Schema zum Lösen von  $f(n) = s(n) + c * f(n-1)$}:  1.c in $f(n) = d * c^{n-1} + p(n)$ eisetzen.\\
 2.Spezielle Lösung p(n) bestimmen. und mögliche Param.x,y,.. durch einsetzen von p(n) bestimmen. (achtung Warnung! Wenn bereits Lösung hom. Rek.gleichung)\\
 3. d durch Einsetzen der Anfangsbeding. bestimmen.\\
 
\subsection{Homogen, lin. Rek., 2. Ordnung}
-Resultat ist eine quadr. Gleichung, die wir mit der Mittenachtsformel lösen können.
\textbf{Charakteristische Gleichung} Die Gleichung : $ax^2+bx+c = 0$ heisst. char. Gleich. für die Rekursionsgleichung: $f(n) = b * f(n-1) + c* f(n-2)\\
Lösung dann über Mitternachtsformel.\\
-für b^2 + 4c > 0: f(n) = d_1 * ....
\subsection{Inhomogen, lin. Rek., 2. Ordnung}...
\pagebreak



