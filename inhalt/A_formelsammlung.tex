
\fbox{\parbox{\columnwidth}{
\section{Formelsammlung}
\subsection*{Binomische Formeln}
$(a+b)^2 = a^2 + 2ab + b^2$\\
$(a-b)^2 = a^2 - 2ab + b^2$\\
$(a+b) \cdot (a-b) = a^2 - b^2$\\
$(a \pm b)^3 = a^3 \pm 3 a^2 b + 3 a b^2 \pm b^3$\\
$(a \pm b)^4 = a^4 \pm 4 a^3 b + 6 a^2 b^2 \pm 4 a b^3 + b^4$\\
$(a \pm b)^5 = a^5 \pm 5 a^4 b + 10 a^3 b^2 \pm 10 a^2 b^3 + 5 a b^4 \pm b^5$
\subsection*{Potenzgesetze}
$a^{-n}=\frac{1}{a^n}$\\
$a^m\cdot a^n=a^{m+n}$\\
$\frac{a^m}{a^n}=a^{m-n}$\\
$(a^m)^n=a^{m\cdot n}$\\
$a^n\cdot b^n=(a\cdot b)^n$\\
$\frac{a^n}{b^n}=(\frac{a}{b})^n$
\subsection*{Wurzelgesetze}
$\sqrt[n]{a}=a^{\frac{1}{n}}$\\
$\sqrt[n]{a^m}=(\sqrt[n]{a})^m=a^{\frac{m}{n}}$\\
$\sqrt[n]{a}\cdot\sqrt[n]{b}=\sqrt[n]{a\cdot b}$\\
$\frac{\sqrt[n]{a}}{\sqrt[n]{b}}=\sqrt[n]{\frac{a}{b}}$\\
$\sqrt[n]{\sqrt[m]{a}}=\sqrt[{n\cdot m}]{a}$
\subsection*{Summeneigenschaften}
%$\sum_{i=1}^n c=n\cdot c$\\
%$\sum_{i=m}^n c=(n-m+1)\cdot c$\\
$\sum_{i=m}^n c\cdot a_i=c\cdot \sum_{i=m}^n a_i$\\
$\sum_{i=m}^n (a_i+b_i)=\sum_{i=m}^n a_i + \sum_{i=m}^n b_i$
\subsection*{Summenformeln}
Gaussche Summenformel:\\
$\sum_{i=1}^n i=\frac{n(n+1)}{2}$\\
Geometrische Reihe:\\
$\sum_{i=1}^n q^i=\frac{1-q^{n+1}}{1-q}$\\
Potenzsummen:\\
$\sum_{i=1}^n i^2=\frac{n(n+1)(2n+1)}{6}$\\
$\sum_{i=1}^n i^3=\frac{n^2(n+1)^2}{4}$\\
\subsection*{Mitternachtsformel}
ax^2+bx+c = 0\\
x_{1/2} = \frac{-b \pm \sqrt{b^2-4ac}}{2a}

\subsection*{Definitionen}
Ungerade Zahlen: für k $\in \mathbb{N}_0$: 2k+1.\\
}}