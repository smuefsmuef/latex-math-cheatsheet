\scriptsize	
\section{Aussagelogik}
\subsection{Aussagen und aussagelogische Formeln}

Eine \textbf{\textit{logische Aussage}} ist eine Aussage, die einen eindeutigen Wahrheitswert hat und somit grunsätzlich entscheidbar ist.
Einzelne entscheidbare Aussagen werden auch \textbf{\textit{elementare oder atomare Aussagen}} genannt.\\

	\textit{V$_{ar}$} &:= Ansammlung\ von\ atom.\ Aussagen. \\
	f &:= Formel\ der\ Aussagelogik\ ueber\ \textit{V$_{ar}$}. \\
	\textit{F$_{AL}$} &:= Gesamtheit\ aller\ aussagelog.\ Formeln. \\
		\textit{V$_{ar}$}(f) &:= Gesamtheit\ aller\ atomaren\ Formeln. \\
			Literal &:= Atomare\ Formel\ oder\ deren\ Negation. \\

\subsection{Aussagelogische Verknüpfungen}
    \textbf{Konjunktion, 'und' } $\land \\
    $
     \[
\begin {array}{c | c || c c } 
    A & B & A \land B \\ \hline
    0 & 0 & 0 \\
    0 & 1 & 0  \\
    1 & 0 & 0 \\
    1 & 1 & 1 
\end{array}
\]
  

   \textbf{Disjunktion, 'oder' } $\vee \\
$ 
    \[
\begin{array}{c | c || c c l}
    A & B & A \vee B &&  \\ \hline
    0 & 0 & 0 &&  \\
    0 & 1 & 1  && \\
    1 & 0 & 1  &&  \\
    1 & 1 & 1  && * \\
\end{array}
\]
 *Aussage wahrim Gegensatz zum umgangsprachlichen 'entweder oder'. Mein\ Name\ ist\ Petra\ oder\ die\ Wand\ ist\ weiss.

    \item\textbf{Negation, 'nicht A' } $\neg \\
$
     \[
\begin{array}{c || c c l}
    A & \neg A && Beispiele\\ \hline
    0 & 1 && Die\ Wand\ ist\ nicht\ rot. \\
    1 & 0 && Die\ Wand\ ist\ nicht\ weiss. \\
\end{array}
\]


   \textbf{Äquivalenz, 'A genau dann, wenn B...' } $\leftrightarrow	
$ 
    \[
\begin{array}{c | c || c c}
    A & B & A \leftrightarrow B \\ \hline
    0 & 0 & 1 \\
    0 & 1 & 0 \\
    1 & 0 & 0 \\
    1 & 1 & 1 \\
\end{array}
\]
Bsp.Mein\ Name\ ist\ Max\ genau\ dann,\ wenn\ die\ Wand\ rot\ ist.\\
 Beachte: Wenn beide Seiten gleich sind, ist die Schlussfolgerung wahr.


   \item\textbf{Implikation, 'Wenn A, dann B', 'Aus A folgt B' } $\rightarrow	\\
  $Prämisse  $\rightarrow \ $Konklusion.
   
    \[
\begin{array}{c | c || c c}
    A & B & A \rightarrow B \\ \hline
    0 & 0 & 1 \\
    0 & 1 & 1 \\
    1 & 0 & 0 \\
    1 & 1 & 1 \\
\end{array}
\]
 Beachte: Wenn A falsch ist, dann ist die Schlussfolgerung wahr. \\

\textbf{\textit{Bindungsprioritäten}} :
1. Negation (bindet am stärksten)
2. Konjunktion / Disjunktion
3.Äquivalenz / Implikation

 \textbf{\textit{Semantisch gleiche}} Formeln $f \equiv g$. = exakt gleiche Wahrheitswerte in Tabellen.\\
 \textbf{\textit{Syntaktisch gleich}} aussagelogischen Formeln $f = g$. = exakt gleichen Folge von Zeichen/Symbolen bestehen.\\
 \textbf{\textit{Ableitungsbaum}} (auch Syntaxbaum, Parsebaum) zeigt Syntax. \\

\subsection{Rechenregeln für aussagelogische Formeln}
z.B. durch Wahrheitstabellen beweisen.
\fbox{\parbox{\columnwidth}{
\textbf{Tautologie und Kontradiktion}
 'konstant'\\
$	true$ &:= A \vee \neg A \ Tautologie. \\
	false &:= A \land \neg A \ Kontradiktion. 
	
\textbf{Idempotenzgesetz}\\
$	f$ \land f \equiv f   \\   f \vee f \equiv f 

\textbf{Kommutativgesetz}\\
$	f$ \land g \equiv g \land f \\ f \vee g \equiv g \vee f

\textbf{Identitätsgesetz}\\ 
$f$ \land true \equiv f\\          	f \vee false \equiv f\\
	f$ \land false \equiv false   \\   	f \vee true \equiv true

\textbf{Assoziativgesetz}\\
	($f$ \land g) \land h \equiv f \land (g\land h) \\
	  	(f \vee g) \vee h \equiv f \vee (g\vee h)\\

\textbf{Absorptionsgesetz}\\
	$f$ \land (f \vee g) \equiv f  \\   f \vee (f \land g) \equiv 

\textbf{Distributivitätsgesetz}\\
	$f$ \land (f \vee h) \equiv (f \land g) \vee (f \land h)\\
	f \vee (f \land h) \equiv (f \vee g) \land (f \vee h)

\textbf{De Morgan Gesetz}\\
	\neg(f \land g) \equiv \neg f \vee \neg g \\  	
	\neg(f \vee g) \equiv \neg f \land \neg g

\textbf{Negationsgesetz}\\
	\neg(\neg f) \equiv f

 \textbf{Idempotenzgesetz}\\
$(p \wedge p) \equiv p$\\
$(p \vee p) \equiv p$
  
  \textbf{Absorptionsregeln}\\
$(p \wedge (p \vee q)) \equiv p$\\
$(p \vee (p \wedge q)) \equiv p$
}}
$



\subsection{Normalformen}

     \textbf{Konjunktive Normalform}\\
für m $\ge$ 1, \textit{L$_{i}$} = Literale.
\begin{aligned}
f = {{\bigwedge_{i=1}^n} {\bigvee_{j=1}^{m_i}}} L_{i_j}
\end{aligned}


  \textbf{Disjunktive Normalform(Momon)}\\
\begin{aligned}
f = {{\bigvee_{i=1} ^n} {\bigwedge_{j=1}^{m_i}}} L_{i_j}
\end{aligned}

$
\subsection{Belegung}
Die \textbf{\textit{Belegung A(f)}} der aussagelogischen Formel \textit{f} ist eine Belegung der Menge \textit{V$_{ar}$()}.\\
	(\textit{V$_{ar} ()$} := Mehrere atom. Formeln (A, B, C) einer aussagelog. Formel \textit{f}.) 
\begin{aligned}
	z &:= (A \rightarrow (B \vee \neg C))., \textit{V$_{ar}$}(z) &:= A, B, C. \\
\end{aligned}


Die Belegung \textit{${A_1}$}(z) ist \textbf{\textit{wahrmachend}} (oder \textbf{\textit{nicht wahrmachend}} )und somit \textbf{\textit{Modell}} für \textit{f}, wenn:\
\begin{aligned}
A \rightarrow true, B \rightarrow false, C \rightarrow false\\
	(true \rightarrow (false \vee true) \equiv true), &&  =	\textit{${A_1}$}(z) = true.
\end{aligned}

$


\subsection{Erfüllbarkeitseigenschaften}
 \textbf{\textit{Erfüllbarkeitsproblem}} (eng. satisfiability, SAT) = aus.log. Form. erfüllbar?\\
 \textbf{\textit{Tautologieproblem (TAUT)}} = aus.log. Form. Tautologie?.\\

\textbf{Erüllbar}\\
mind. eine wahrmachende Belegung ()\textbf{\textit{mind. ein Modell}}).
\begin{aligned}
A := Heute\ ist\ ein\ Herbsttag. \\ \textit{f} := \neg A = Heute\ ist\ nicht\ ein\ Herbsttag.
\end{aligned}

 \[
\begin{array}{c || c }
    A & f=\neg A\\ \hline
    0 & 1  \\
    1 & 0 \\
\end{array}
\]



\textbf{Unerfüllbar}\\
Keine wahrmachende Belegung, \textbf{\textit{kein Modell}}.
\begin{aligned}
A := Jetzt\ ist\ Mittagszeit. \\ \textit{t} := A \land \neg A = Jetzt\ ist\ Mittagszeit\ und\\ jetzt\ ist\ nicht\ Mittagszeit.\\
\end{aligned}

 \[
\begin{array}{c || c }
    A & t=A \land \neg A\\ \hline
    0 & 0  \\
    1 & 0 \\
\end{array}
\]


\textbf{Tautologie}\\
 \textbf{\textit{Nur wahrmachende Belegungen}}. Alles Modelle.
\begin{aligned}
A := Heute\ ist\ ein\ Herbsttag. \\ \textit{y} := A \vee \neg A = Heute\ ist\ ein\ Herbsttag\\ oder\ Heute\ ist\ nicht\ ein\ Herbsttag.
\end{aligned}

 \[
\begin{array}{c || c }
    A & y=A \vee \neg A\\ \hline
    0 & 1  \\
    1 & 1 \\
\end{array}
\]

\textbf{Falsifizierbar}\\
 \textbf{\textit{Mind. eine falschmachende}} Belegung, d.h. mind. eine Belegung ist kein Modell.
\begin{aligned}
A := Heute\ ist\ ein\ Herbsttag. , B := Jetzt\ ist\ Mittag. \\ 
\textit{p} := A \land B = Heute\ ist\ ein\ Herbsttag\\ und\ jetzt\ ist\ Mittagszeit.
\end{aligned}

 \[
\begin{array}{c | c || c }
    A & b & p=A \land B\\ \hline
    0 & 0 & 0\\
    0 & 1 & 0\\
    1 & 0 & 0\\
    1 & 1 & 1
\end{array}
\]

\textbf{Neutralität / Kontingenz}\\
 \textbf{\textit{ weder Tautologie noch eine Kontradiktion}}
$$

\section{Prädikat und Quantoren}
\textbf{\textit{Eigenschaft}} eines untersuchten Objektes. \\
\subsection{Prädikat und prädikatenlogische Formeln}
P(x) ist eine \textbf{\textit{Aussageform}} und  keine Aussage = kein Wahrheitswert. Erst mit  \textbf{\textit{Belegung}} zur Aussage.
\begin{aligned}
P(x)  := x\ ist\ eine \ Pflanze.\\ 
\end{aligned}
wenn nun x eine Begonie ist, dann ist P(Begonie) wahr. Wenn x eine Krabbe ist, dann ist P(Krabbe) falsch.

T(x) ist \textbf{\textit{äquivalent}} zu W(x), wenn beide Prädikate für alle Belegungen die selben Wahrheitswerte aufweisen. Quantor (bindet am stärksten), danach $\vee$ , $\land$ u. a.


\subsection{Quantoren}
\textbf{Allquantor $\forall$} \textbf{\textit{für alle x gilt P(x)}}.
\begin{aligned}
\forall	x : P(x)
\end{aligned}

\textbf{Existenzquantor $\exists$} \textbf{\textit{es (mind.) 1 x gibt, so dass P(x) gilt}}.
\begin{aligned}
\exists x : P(x)
\end{aligned}

\textbf{Eindeutigkeitsquantor $\exists!$} \textbf{\textit{es gibt genau ein x, so dass P(x) gilt}}.
\begin{aligned}
\exists! x : P(x)
\end{aligned}

\fbox{\parbox{\columnwidth}{
\textbf{Verneinungssätze}\\
\neg (\forall x : A(x)) \leftrightarrow \exists x : \neg A (x)\\
\neg (\forall x : \neg A(x)) \leftrightarrow \exists x : A (x)\\
\neg (\exists x : A(x)) \leftrightarrow \forall x : \neg A (x)\\
\neg (\exists x : \neg A(x)) \leftrightarrow \forall x : A (x)


\textbf{Vertauschbarkeitssätze}\\
\forall x,y : A(x,y) \leftrightarrow \forall x,y : A(x,y)\\
\exists x,y : A(x,y) \leftrightarrow \exists x,y : A(x,y)\\
\exists x \forall y : A(x,y) \leftrightarrow \forall y \exists x : A(x,y)\\
\forall x : A(x) \rightarrow \exists x : A(x)\\
\exists! x : A(x) \rightarrow \exists x : A(x)
}
}}

\subsection*{Äquivalenzumformung}
Wir demonstrieren an der Formel $\neg (\neg p \wedge q) \wedge (p \vee q)$, wie man mit Hilfe der
aufgelisteten logischen Äquivalenzen tatsächlich zu Vereinfachungen kommen kann:\\
$\phantom{{}\equiv{}} \neg (\neg p \wedge q) \wedge (p \vee q)$\\
$\equiv (\neg (\neg p) \vee (\neg q)) \wedge (p \vee q)$\hfill de Morgan\\
$\equiv (p \vee (\neg q)) \wedge (p \vee q)$\hfill Doppelnegation\\
$\equiv p \vee ((\neg q) \wedge q)$\hfill Distributivtät v.r.n.l.\\
$\equiv p \vee (q \wedge (\neg q))$\hfill Kommutativtät\\
$\equiv p \vee f$\hfill Prinzip v. ausgeschl. Widerspruch\\
$\equiv p$\hfill Kontradiktionsregel\\